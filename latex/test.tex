\documentclass[a4paper,12pt,twocolumn]{book}

\title{\textbf{ex vinum}}
\author{by tsbohc}
\date{}

\usepackage{blindtext}
\usepackage[utf8]{inputenc}
\usepackage[russian]{babel}
\usepackage{titlesec}
\usepackage{epigraph}
\usepackage{lettrine}
\usepackage{indentfirst}

\setlength\epigraphrule{0pt}

\titlespacing\chapter{0pt}{12pt plus 4pt minus 2pt}{12pt plus 2pt minus 2pt}
\titleclass{\chapter}{straight}

\titleformat{\chapter}
  {\large\bfseries} % format
  {}                % label
  {0pt}             % sep
  {\LARGE\filcenter}           % before-code

\titleformat{\section}
  {\large\bfseries} % format
  {}                % label
  {0pt}             % sep
  {\large}          % before-code

\setlength{\DefaultNindent}{0pt}
\setlength{\DefaultFindent}{3pt}

\begin{document}

\maketitle
\tableofcontents

\chapter{введение}

\epigraph{Да раскинутся границы твоего аутизма до дальних цветочных полей и затянутых дымкой горных вершин.}{--- неизвестный волшебник}

\lettrine[lines=2]{E}{x vinum} -- настольная ролевая игра, в сердце которой лежат любовь к рпг и аутизм, границы между которыми были размыты вином. Так она и получила свое название.

Игра делает акцент на ролевом аспекте, упрощая но при этом не отказываясь от признанных систем.

Мир игры -- классический фэнтези в стиле D\&D, не претендующий на строгий реализм.

\section{общие положения}

DM и игроки работают вместе, чтобы создать общий нарратив, а не играют друг против друга.

Что бы ни было написано в правилах, DM всегда имеет последнее слово.

DM должен поощрять изобретательность игроков, даже если их идеи не реалистичны.

\section{процесс игры}

DM описывает ситуацию. Детали окружения, персонажей и так далее. Однако, далеко не вся информация доступна игрокам сразу.

Игроки планируют свои действия, задают вопросы. Приняв решения, оглашают их, а затем бросают кости. Успешность действий зависит от сложности их выполнения и аттрибутов соотвествующих персонажей.

DM описывает результаты действий игроков.

\chapter{создание персонажа}

\lettrine{С}{}оздание персонажа проходит в два этапа: подготовки и непосредственно перед началом игры.

\section{подготовка}

Пояснения ниже служат вдохновением -- не каждому персонажу необходима ковбойская шляпа, но каждый персонаж имеет какую-то внешность.

\begin{enumerate}
  \item \textbf{Имя, пол, возраст}. Фамилия, псевдоним, прозвище. Возраст должен быть согласован с расой.
  \item \textbf{Внешность}. Телосложение, лицо, одежда, головной убор, шрамы, татуировки...
  \item \textbf{Предыстория}. Как ваш персонаж стал тем, кто он сейчас? Что привело вас к началу кампании? Как вы получили тот шрам или потеряли своего старого друга? Есть ли у вас возлюбленная или возлюбленный?
  \item \textbf{Личность}. Во что верит ваш персонаж? Как он ведет себя в отношении других людей? Как хорошо справляется с критическими ситуациями?
\end{enumerate}

\section{за столом}

Лучший персонаж получает +1 к выбранному аттрибуту

\end{document}

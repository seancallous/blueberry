\documentclass[a4paper,12pt,twocolumn]{book}

\title{\textbf{EX VINUM}}
\author{by tsbohc}
\date{}

\usepackage[table]{xcolor}
\usepackage{hyperref}
\usepackage{blindtext}
\usepackage[utf8]{inputenc}
\usepackage[russian]{babel}
\usepackage{titlesec}
\usepackage{epigraph}
\usepackage{lettrine}
\usepackage{indentfirst}
 
\setlength\epigraphrule{0pt}

\titlespacing\chapter{0pt}{12pt plus 4pt minus 2pt}{12pt plus 2pt minus 2pt}
\titleclass{\chapter}{straight}

\titleformat{\chapter}
  {\large\bfseries} % format
  {}                % label
  {0pt}             % sep
  {\LARGE\filcenter}           % before-code

\titleformat{\section}
  {\large\bfseries} % format
  {}                % label
  {0pt}             % sep
  {\large}          % before-code

\setlength{\DefaultNindent}{0pt}
\setlength{\DefaultFindent}{3pt}

\begin{document}

\maketitle
\tableofcontents

\chapter{Введение}
\epigraph{Да раскинутся границы твоего аутизма до дальних цветочных полей и затянутых дымкой горных вершин.}{--- неизвестный волшебник}

\lettrine[lines=2]{E}{x vinum} -- настольная ролевая игра, в сердце которой лежат любовь к рпг и аутизм, границы между которыми были размыты вином. Так она и получила свое название.

Игра делает акцент на ролевом аспекте, упрощая но при этом не отказываясь от признанных систем.

Мир игры -- классический фэнтези в стиле D\&D, не претендующий на строгий реализм.


\section{Процесс Игры}
Dungeon Master описывает ситуацию. Детали окружения, персонажей. Однако, далеко не вся информация доступна игрокам сразу.

Игроки планируют свои действия, задают вопросы. Приняв решения, оглашают их, а затем бросают кости. Успешность действий зависит от сложности их выполнения и аттрибутов соотвествующих персонажей.

DM описывает результаты действий игроков.


\section{Nota Bene}
DM и игроки работают вместе, чтобы создать общий нарратив, а не играют друг против друга.

Что бы ни было написано в правилах, DM всегда имеет последнее слово.

DM должен поощрять изобретательность игроков, даже если их идеи не реалистичны.

DM должен поощрять способность игроков вживаться в роли.

Главная цель не победить, а хорошо провести время в кругу друзей.

\chapter{Ролевая Система}

\lettrine{В}{}основе игры лежит d12. Такой выбор обусловлен желанием привнести больше динамики в традиционную систему d20.

Персонажи обладают всего тремя аттрибутами, каждый из которых имеет второстепенный эффект.


\begin{enumerate}
  \item \textbf{Strength} -- Constitution

  определяет физическую силу и выносливость, а также имеет большое влияние на общий запас здоровья.

  \item \textbf{Dexterity} -- Perception

  определяет ловкость и скорость реакции, а также внимательность, слабо влияет на запас здоровья.

\item \textbf{Intelligence} -- Charisma

  определяет ум и сообразительность, фоновые знания, а также уменее вести разговор, не влияет на запас здоровья.
\end{enumerate}

\pagebreak

\section{Расы}


\begin{enumerate}
  \item \textbf{Human}.

  +1 любому аттрибуту

  \item \textbf{Dwarf}.

  +2 str, -1 dex

  \item \textbf{Elf}.

  +2 int, -1 str

  \item \textbf{Tiefling}.

  +2 dex, -1 int

  \item \textbf{Druergar}.

  +1 str, +1 dex, -1 int

  \item \textbf{Drow}.

  -1 str, +1 dex, +1 int

  \item \textbf{Aasimar}.

  +1 str, -1 dex, +1 int

\end{enumerate}

\section{Attribute Modifiers}

Бонусы спользуются чтобы подчеркнуть, насколько хорошо или плохо персонаж владеет тем или иным навыком.

Высокие значения аттрибутов пораждают положительные бонусы, низкие -- отрицательные.

Расчитать свои бонусы можно по формуле или таблице:

\[attr\ bonus = floor((attr - 6.5)/2)\]

\begin{center}
\definecolor{lightgray}{gray}{0.95}
\rowcolors{1}{}{lightgray}
\begin{tabular}{ c c | c c }
  \emph{attribute} & \emph{mod} & \emph{attribute} & \emph{mod} \\
  \hline
  1-2 & --3 & 9-10 & +1 \\
  3-4 & --2 & 11-12 & +2 \\
  5-6 & --1 & 13-14 & +3 \\
  7-8 & --0 & 15-16 & +4

\end{tabular}
\end{center}

\section{Ability Checks}

В случае, когда действие игрока имеет шанс на провал, DM просит его сделать ролл на проверку аттрибута.

Успешность действия определяется сложением ролла d12 и соответствующего бонуса.

Для успешного выполнения действия, итоговый ролл должен быть равен или превосходить сложность установленную DM. Сложность при этом не разглашается, только результат.

Другими словами, действие успешно если:

\[d12 + attr\ bonus \geq difficulty \]


\chapter{Создание Персонажа}

\lettrine{С}{}оздание персонажа проходит в два этапа: подготовки и непосредственно перед началом игры.

\begin{enumerate}
  \item \textbf{Имя, пол, возраст} (фамилия, псевдоним, прозвище...)
  \item \textbf{Внешность}. Телосложение, лицо, одежда, головной убор, шрамы, татуировки...
  \item \textbf{Предыстория}. Как ваш персонаж стал тем, кто он сейчас? Что привело вас к началу кампании? Как вы получили тот шрам или потеряли своего старого друга? Есть ли у вас возлюбленная или возлюбленный?
  \item \textbf{Личность}. Во что верит ваш персонаж? Как он ведет себя в отношении других людей? Как хорошо справляется с критическими ситуациями?
\end{enumerate}

Лучший персонаж получает +1 к выбранному аттрибуту





\pagebreak
\chapter{практические примеры}

\hypertarget{abilitycheckexample}{\section{ability check}}

Персонаж Josh довольно слабый, у него всего 4 strength:

\*

\(str\ b = floor((4 - 7)/2)\)

\(str\ b = floor(-3/2)\)

\(str\ b = floor(-1.5)\)

\*

\emph{floor} говорит о том, что полученное значение округляется вниз. Таким образом:

\[strength\ bonus = -2\]

Представим ситуацию, что Josh хочет пробить плечем дверь. DM просит игрока сделать ability check на strength. Игрок берет d12 и выбрасывает 7. Обращаясь к формуле выше, к полученному роллу он прибавляет свой бонус:

\[7 + (-2) = 5\]

Допустим, что дверь совсем новая, и прочно укреплена. DM выставил сложность действия 6 (среднея). Итоговый ролл \emph{y} меньше данного числа, и дверь не поддается. 

В противном случае, представим что дверь ветхая и давно прогнила. DM выставил сложность 3 (низкая) и Josh пробивает ее не смотря на свою невысокую силу.
\end{document}
